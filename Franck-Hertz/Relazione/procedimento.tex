\section{Procedura operativa}

	Si è acceso il tetrodo regolando la tensione del catodo $U_F$ = \SI{8.0 \pm 0.1}{\volt}
	in modo da innescare l'emissione di $e^{-}$ per effetto termoionico.

	In una prima fase si è regolata la tensione sulla griglia d'anodo imponendo $U_A=$ \SI{70 \pm 0.1}{\volt},	dopodichè si è modificata la tensione $U_G$ sulla griglia di controllo aumentandola fino alla comparsa di tre bande luminose arancioni.
	La griglia di controllo ha una duplice funzione:
	\begin{itemize}
	\item  stabilire un campo elettrico sulla superficie del catodo favorendo l'emissione di elettroni e aumentando di fatti la luminosità  delle bande;
	\item trattandosi di una fitta rete di fili sottili permette il passaggio degli elettroni e fissandone il potenziale permette di generare un campo uniforme fino alla griglia d'anodo (come in un condensatore a facce piane e parallele); al diminuire della tensione $U_G$ si nota infatti un cambiamento nella forma delle bande luminose (si distorcono ai bordi) venendo a mancare l'uniformità  del campo.
	\end{itemize}

	Gli elettroni risultano accelerati dalla d.d.p. $U_A-U_G$.

\paragraph{}
	Si è osservato che aumentando la tensione $U_A$ partendo da $\SI{0}{\volt}$ si ottiene
	dapprima la comparsa di una banda luminosa nella zona della griglia d'anodo, quindi lo spostamento della riga verso il catodo e successivamente, continuando ad aumentare  la tensione, la comparsa di un'altra banda sull'anodo, che evolve come la precedente.
	 Nel range di tensioni permesse dall'apparato è possibile vedere contemporaneamente 3 bande.

		Questo andamento è da imputarsi al fatto che il trasferimento
		di energia negli urti tra $e^{-}$ e atomi di Ne sia
		quantizzato: un urto anelastico è possibile esclusivamente quando hanno gli elettroni hanno energia maggiore o uguale
		al gap energetico tra primi livelli eccitati liberi.

		Poichè l'energia degli $e^{-}$ è proporzionale alla distanza
		percorsa dalla griglia di controllo ed alla d.d.p. $U_A-U_G$,
		all'aumentare di $U_A$ si riduce
		la distanza necessaria affinchè gli $e^-$ abbiano l'energia sufficiente
		ad eccitare gli atomi di Ne, pertanto la banda si sposta verso il catodo.

		All'ulteriore aumento del campo accelerante dopo il
		primo urto, e conseguentemente all'eccitazione degli
		atomi di neon, gli elettroni acquisiscono un energia
		sufficiente per rieccitare gli atomi
		di neon in un tratto successivo del tetrodo ;
		a ciò corrisponde la comparsa delle ulteriori righe colorate.
\paragraph{}
	Si è proceduto alla misura della d.d.p. $U_A-U_G$ di apparizione delle bande luminose. Ogni membro del gruppo ha ripetuto la misura, il risultato della media è di seguito riportato:

	\begin{table}[H]
 		\centering
		\begin{tabular}{cc}
 			\toprule
 			$\#$  &  $U_A- U_G$\\
  			\midrule
  			$1$ & \SI{15.6 \pm 1.0}{\volt}\\
  			$2$ &  \SI{32.1 \pm 1.0}{\volt}\\
  			$3$ & \SI{53.5 \pm 1.0}{\volt}\\
  			\bottomrule
 		\end{tabular}
	\label{tab:a}
	\end{table}



	Ci si aspetta, almeno in prima approssimazione, che le differenze fra le d.d.p $U_A-U_G$ tra due bande successive sia la differenza di energia tra i primi livelli eccitati liberi degli atomi di Neon.
	In particolare si ottiene $\Delta_{12} = \SI{16.5 \pm 1.4}{\volt}$ e $\Delta_{23} = \SI{21.4 \pm 1.4}{\volt}$, che sono da confrontarsi con la differenze di energia dei primi livelli eccitati del Neon rispetto al fondamentale ($1s^2 2s^2 2p^6$):
	$$E_1 \simeq \SI{16.7}{\eV} \quad (1s^2 2s^2 2p^5 3s) \qquad \text{ e } \qquad E_2 \simeq \SI{18.6}{\eV} \quad (1s^2 2s^2 2p^5 3p).$$

\subsection{}
	Si è settato l'apparato strumentale in modo che
	la tensione $U_A$ sia in modalità ramp: varierà  automaticamente tra $\SI{0}{\volt}$ e il massimo permesso dall'apparato:
	(\SI{80.0 \pm 0.5}{\volt}). La tensione della griglia di controllo è invece lasciata fissa al valore
	$U_G = \SI{3.5 \pm 0.1}{\V}$

Il collettore è collegato al generatore di tensione mediante un amplificatore a transimpedenza
	che fornisce in uscita un segnale in tensione proporzionale alla corrente di collettore $I_C$.

	Attraverso la modalità  X-Y dell'oscilloscopio si è quindi
	osservato l'andamento della curva $I_C \text{ vs } U_A$ al
	variare di $U_E$, la tensione del collettore.

	L'applicazione di una d.d.p tra griglia d'anodo e collettore ha lo scopo di creare un campo elettrico frenante per gli elettroni che emergono dall'anodo verso il collettore.

	In \figurename{ \ref{fig:ue}} sono riportati gli andamenti osservati.

\begin{figure}[h!]
		\centering
		 \subfloat[$U_E=\SI{1.0 \pm 0.1}{\volt}$]{
		\includegraphics[scale=0.6]{../Figs-tabs/ue3.png}
		\label{fig:ue1}
		}
		 \subfloat[$U_E= \SI{3.4 \pm 0.1}{\volt}$]{
		\includegraphics[scale=0.6]{../Figs-tabs/ue34.png}
		\label{fig:ue3.4}
		}
		 \subfloat[$U_E= \SI{7.5 \pm 0.6}{\volt}$]{
		\includegraphics[scale=0.6]{../Figs-tabs/ue75.png}
		\label{fig:ue7.5}
		}
		\caption{$I_C$ vs $U_A$ al variare di $U_E$ in modalità  X-Y}
	\label{fig:ue}
\end{figure}
	Come è possibile osservare da \figurename{ \ref{fig:ue1}} e \figurename{ \ref{fig:ue3.4}}
	per valori bassi del campo decelerante $U_E$ la quasi totalità  degli $e^{-}$, anche qualora facciano
	urti anelastici, raggiunge il collettore.
	Aumentando $U_E$ si osserva un decremento
	di $I_C$, ciò risulta compatibile col fatto che gli $e^{-}$
	dopo aver ceduto energia per eccitare gli atomi di He
	non hanno energia sufficiente a superare il campo di decelerazione.

In \figurename{ \ref{fig:ue7.5}} è evidente il classico andamento  : all'aumentare di $U_A$ corrisponde un aumento di $I_C$ fino ad un massimo, che coincide col momento in cui gli $e^-$ iniziano ad avere energia sufficiente per eccitare gli atomi di Neon, a questo evento corrisponde la comparsa della prima banda.

	Il minimo che segue è nel punto in cui la banda si stacca dalla griglia d'anodo: in questo caso la maggior parte degli elettroni supera la griglia con energia quasi nulla: vengono fermati dal campo frenante, non arrivando quindi sul collettore.
	All'aumentare della tensione $U_A$ la corrente torna ad aumentare fino a quando gli elettroni non avranno nuovamente l'energia necessaria ad eccitare il Ne.

	Dalla regolazione di $U_E$ si sono settati i minimi della curva
	$I_C VS U_A$ in corrispondenza dello zero.
	Si è regolato il valore di $U_A$ in maniera da osservare il maggior
	numero di massimi osservabili sullo schermo dell'oscilloscopio,
	è stato inoltre necessario regolare il guadagno dell'amplificatore
	in maniera da non saturare l'OpAmp.

	A seguito si sono acquisite le due tracce di $U_A$ e $\propto I_C$ al variare di
	$U_E$ da $\SI{9}{\volt}$ a $\SI{0}{\volt}$. Per  tali acquisizioni, qualora non si siano
	verificati problemi a rilevare i picchi, sono stati
	osservate le $U_A$ corrispondenti ai massimi di $I_C$.

\sisetup{table-figures-decimal = 1, table-figures-exponent = 0, table-figures-integer = 2, table-figures-uncertainty = 1}

	\begin{table}[H]
		\centering
		\begin{tabular}{SSSSS}
			\toprule
			{\multirow{2}{*}{$U_E\;[\si{\volt}]$}}  &  \multicolumn{4}{c}{$U_A-U_G\;[\si{\volt}]$}\\
			 &	{1} & {2} & {3} & {4}\\
			\midrule
			9.6(1) & 15.2(8) & 33.4(8) & 53.3(8) & 73.2(8)\\
			9.0(1) & 15.6(8) & 32.5(8) & 52.9(8) & 73.2(8)\\
			8.0(1) & 14.3(8) & 32.5(8) & 52.5(8) & 72.8(8)\\
			6.9(1) & 15.2(8) & 32.5(8) & 53.3(8) &\\
			6.0(1) & 14.7(8) & 32.1(8) & &\\
			5.0(1) & 13.5(8) & 31.7(8) & &\\
			4.0(1) & 12.6(8) & & &\\
			3.0(1) & 13.0(8) & & &\\
			\bottomrule
		\end{tabular}
		\label{tab:b}
	\end{table}
Le caselle vuote corrispondo a massimi non misurabili.
Si è poi proceduto ad un fit lineare, ottenendo i seguenti risultati:
	\begin{table}[H]
		\centering
		\begin{tabular}{ccc}
			$\Delta V_1 = \SI{14.3 \pm 0.3}{\volt}$ && $\chi^2/ndof = 13/7$\\
			$\Delta V_2 = \SI{32.5 \pm 0.3}{\volt}$ && $\chi^2/ndof = 2.5/5$\\
			$\Delta V_3 = \SI{53.0 \pm 0.4}{\volt}$ && $\chi^2/ndof = 0.8/3$\\
			$\Delta V_4 = \SI{73.1 \pm 0.5}{\volt}$ && $\chi^2/ndof = 0.2/2$\\
		\end{tabular}
	\end{table}
	Si riporta il fit in \figurename{ \ref{fit}}.

		\begin{figure} [!h]
			\centering
			\includegraphics[width=0.9\textwidth]{../Figs-tabs/fit.pdf}
			\caption{Dati raccolti e fit.}
			\label{fit}
		\end{figure}

	Le differenze tra i massimi sono dunque:
	$$\Delta_{12}=\SI{18.2\pm 0.4}{\volt} \qquad \Delta_{23}=\SI{20.5 \pm 0.5}{\volt} \qquad \Delta_{34}=\SI{20.1 \pm 0.6}{\volt}$$
	Si osserva che dati sono in accordo con quelli trovati alla
	verifica qualitativa della comparsa delle frange.

	Si nota come le corrispondenti differenze in enegia degli elettroni non siano
	compatibili con eccitazioni al primo livello eccitato;
	riteniamo che ciò sia dovuto alla non trascurabilità del cammino libero medio
	degli elettroni rispetto alla distanza tra le griglie: un calcolo approssimativo
	in cui determiniamo la densità di particelle dalla legge dei gas perfetti
	(con $T \approx \SI{300}{\K}, p \approx \SI{10}{\milli\bar}$) e utilizziamo
	\SI{0.5}{\angstrom} come raggio del Neon per il calcolo della sezione d'urto
	restituisce infatti un cammino libero medio di $\sim \SI{0.5}{\mm}$,
	ovvero circa un decimo della lunghezza della regione interessata dal campo accelerante.
	Si è eseguito un fit su questi tre punti della relazione tra cammino libero medio,
	energia di eccitazione, ordine del massimo e differenza di energia osservata:
	$$ \Delta E_n = E_n - E_{n-1} = \big(1 + \frac{\lambda}{L}(2n - 1)\big) E_a $$
	È risultato $\frac{\lambda}{L} = \SI{0.05(4)}, \ E_a = \SI{16(3)}{\eV}$, con un $\chi^2$ di 5 (1 \dof).

	L'energia di eccitazione è compatibile con quella nota, ma
	il rapporto tra cammino libero medio e lunghezza della regione di accelerazione
	si conferma non trascurabile, e in particolare sufficiente perché gli elettroni acquisiscano
	energia sufficiente ad eccitare il Neon a livelli più alti, il che
	spiegherebbe anche la poca aderenza delle misure alla curva fittata (ovvero
	l'alto $\chi^2$): staremmo infatti cercando di catalogare quanto osservato
	come la ripetuta eccitazione del Neon ad un livello energetico fissato
	mentre quello che starebbe realmente accadendo è la sovrapposizione di
	eccitazioni a livelli diversi.

\subsection{Note}
Nell'acquisizione dei massimi di $I_C$ contro $U_A$ si è preso come errore
l'incertezza stessa del dato identificato come valore massimo, poiché essa è confrontabile con
la larghezza della regione (pressoché piatta) del massimo e pertanto non riteniamo
di poter sfruttare il numero di dati raccolti per ridurre l'errore in modo sighificativo.

Si sono propagati gli errori sistematici (risultanti da differenze rispetto a valori fissi
o simili operazioni) separatamente rispetto a quelli statistici; in particolare
nell'eseguire i fit si è considerato solo questi ultimi.
