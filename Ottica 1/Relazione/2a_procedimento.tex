\section{svolgimento delle misure}
	La nostra misurazione per essere effettuata si compone di due diverse
	fasi; una fase di 
	calibrazione
	dell'apparato; ed una fase in ci vengono 
	effettuate la misure per determinare $\lambda_{Na}$ 
\subsection{calibrazione }
	La prima fase consiste nell'azzeramento del dispositivo;
	per fare ciò abbiamo rimosso il prisma ed allineato la
	fenditura di uscita della sorgente,la fessura della slitta
	ed il reticolo a croce del telescopio di osservazione.
	Per effettuare tale allineamento sono state impiegate 
	le viti per effettuare spostamenti fini di cui è dotato il piatto
	rotante.
	Abbiamo assunto la lettura del goniometro
	$\alpha_0$ quale zero di riferimento per le successive misure.
	Essendo tale misura basilare per le osservazioni successive
	abbiamo iterato tale misurazione più volte
	ottenendo $\alpha_0=	\pm		$
	Effettuato questo primo step abbiamo reinserito il
	prisma orientandolo in maniera che formi un angolo 
	di $\sim 60$ gradi.
	Si è ruotato il telescopio di osservazione fino ad osservare
	tutte le righe nel visibile emesse dalla nostra sorgente;lampada al 
	cadmio.
	Ruotando a sua volte il prisma si osserva che le righe si spostano;
	si è bloccato il prisma all'angolo di inversione del moto
	della riga più prossima al giallo;ovvero la riga verde.
	Si sono adesso misurate le varie posizioni angolari 
	delle righe di emissione della lampada al cadmio
	\bigskip
	\begin{table}[hb]
		\centering
		\begin{tabular}{|c|c|c|}
 		\hline
		$\lambda [nm]$ & $\alpha\qquad(\alpha_{letto}-\alpha_{0})$ &$\Delta \alpha$ \\
  		\hline
		$467.8$  & 5 &  \\
 		$480.0$ & 19.6 &  \\
 		$508.6$ & 22.2 & \\
 		$643.8$ & 24.4 & \\
 		\hline
 		\end{tabular}
		\caption{Posizione angolari delle frange di 
 		emissione di una lampada al cadmio.Tali misure sono
 		date in riferimento all'$\alpha_0$ precedentemente determinato.\\
  		$\alpha_0=	$.}
		\label{tab:disper_angolare}
	\end{table}
	\bigskip
	.
	Effettuando un grafico della posizione angolare 
	$\alpha$ vs $1/\lambda$ si ottiene una
	dipendenza lineare con un fit a due parametri.
	%\begin{figure} [!h]
	%	\centering
	%	\includegraphics[width=0.9\textwidth]{./fit_angol}
	%	\caption{Fit della posizione angolare delle righe di emissione di 	
	%	una lampada al cadmio rispetto alla $1/\lambda$ di tali righe.}
	%	\label{fig:fit}
	%\end{figure}
	Si riporta il fit in \figura{fig:fit}.
	\smallskip

	I parametri così ottenuti
	\smallskip
	$\text{coefficiente} = a =		\pm		\text{[nm]}
	\qquad \text{intercetta} = b =		\pm		\text{[]}$
	\smallskip
	costituiscono una relazione
	tra la deviazione angolare misurata e la lunghezza d’onda osservata
	anche nella fase
	successiva.
\subsection{procedimento e misure}
	Per effettuare le misure vere e proprie abbiamo 
	rimontato la lampada al sodio.
	Dopo aver aspettato alcuni minuti,ritenuti necessari per la 
	termalizzazione
	della sorgente.
	Abbiamo osservato la posizione angolare $\alpha_{riga gialla Na}=	
	\pm		$
	Dalla relazione della calibrazione spettrale ottenuta attraverso il 
	fit 
	possiamo ricavare 
	\smallskip
	\begin{equation}
		\lambda= \frac{a}{\alpha - b} =		\pm		\text{ [nm]}
	\end{equation}
	a fronte di $ \lambda_{atteso}\sim 589 [nm]$