\documentclass[11pt,a4paper]{article}
\usepackage[T1]{fontenc}
\usepackage[utf8x]{inputenc}
\usepackage[italian]{babel}
\usepackage[italian]{varioref}


\usepackage{amsmath}
\usepackage{amsfonts}
\usepackage{subfigure}
\usepackage{amssymb}
\usepackage{indentfirst}
\usepackage{graphicx}
\usepackage{floatflt}
\usepackage{float}
\usepackage{caption}
\usepackage{tikz}
\usepackage[nomargin,inline,marginclue,draft]{fixme}
\usepackage[left=1.5cm,right=1.5cm,top=1.5cm,bottom=1.5cm]{geometry}
\usepackage{siunitx} %in prova
\newcommand{\rem}[1]{[\emph{#1}]}

%Intestazione
\usepackage{fancyhdr}
\pagestyle{fancy}
\lhead{Esperienza 2}
\chead{Ottica 1}
\rhead{Gruppo BF}

%Dichiarazione di operatori necessari per scrivere formule e unità di misura
\DeclareMathOperator{\uV}{\mu V}
\DeclareMathOperator{\ohm}{\Omega}
\DeclareMathOperator{\kohm}{k\Omega}
\DeclareMathOperator{\Mohm}{M\Omega}
\DeclareMathOperator{\uA}{\mu A}
\DeclareMathOperator{\us}{\mu s}
\DeclareMathOperator{\uF}{\mu F}

%comendi in prova
\newcommand{\figura}[1]{\textsf{Figura \ref{#1}}}
\newcommand{\tabella}[1]{\textsf{Tabella \ref{#1}}}
\newcommand{\equazione}[1]{\textsf{Equazione \ref{#1}}}
\newcommand{\sezione}[1]{\textsf{Sezione \ref{#1}}}

\author{Gruppo BF \\ \smallskip Thomas Giannoni, Valerio Lomanto, Roberto Ribatti}
\title{Esperienza N.2 \\ \smallskip Ottica 2}
\date{24 febbraio 2017}
\begin{document}
\maketitle

\begin{abstract}
In quest'esperienza si è impiegato il fenomeno della diffrazione.

Nella prima sezione per mezzo di uno 
spettroscopio a prisma 
siamo andati a misurare la 
lunghezza d'onda di una riga spettrale
emessa da una lampada al sodio.

Nella seconda parte dell'esperienza abbiamo misurato 
la risoluzione dello spettroscopio a reticolo di diffrazione
in dotazione;dopodiché dall'osservazione delle righe di 
emissione di una lampada all'idrogeno 
si è determinata la costante di Rydberg.

\end{abstract}

%\newpage
\tableofcontents %dovrebbe fare l'indice
\newpage
\part{Misura della lunghezza d'onda della riga gialla del sodio}
\section{Finalità}
	In questa parte dell'esperienza abbiamo determinato la lunghezza d’onda $\lambda_{\text{Na}}$ della riga spettrale
	emessa da una lampada al sodio.
\section{Strumentazione}\label{sez:str_a}
	La strumentazione impiegata in questa sezione è uno spettrometro a
	prisma,riportato in \textbf{figura 1};
	tale spettroscopio si compone di:
	\begin{list}{$\cdot$}{}
		\item \textbf{una sorgente } ovvero una lampada al cadmio in fase
		di
		calibrazione e una lampada al sodio in fase di misura.
		\item \textbf{due telescopi}, uno fisso	,per raccogliere la luce
		della
		sorgente e inviarla sul prisma, ed un telescopio di osservazione
		montato su di un piatto rotante e dotato di goniometro
		(sensibilità di $1/60$ di grado),in grado di ruotare
		rispetto al prisma.
		Il telescopio di raccolta  è munito di una fenditura regolabile
		attraverso la quale regolare l'ingresso della luce.

		Il telescopio di osservazione permette la regolazione del fuoco
		ed è
		inoltre regolabile attraverso delle viti.
		\item \textbf{un prisma} che costituisce l'elemento dispersivo
		dello
		spettroscopio
	\end{list}
	\bigskip
	si è inoltre impiegata una lente di ingrandimento per facilitare la
	lettura della scala del goniometro.

	\bigskip


	\begin{figure} [h]
		\centering
		\includegraphics[width=0.9\textwidth]{./prisma}
		\caption{Schema dell'apparato impiegato.}
		\label{fig:prisma}
	\end{figure}
	Si riporta in \textbf{figura \ref{fig:prisma}} uno schema
	dell'apparato
	sperimentale impiegato.

\section{svolgimento delle misure}
	La nostra misurazione per essere effettuata si compone di due diverse
	fasi; una fase di 
	calibrazione
	dell'apparato; ed una fase in ci vengono 
	effettuate la misure per determinare $\lambda_{Na}$ 
\subsection{calibrazione }
	La prima fase consiste nell'azzeramento del dispositivo;
	per fare ciò abbiamo rimosso il prisma ed allineato la
	fenditura di uscita della sorgente,la fessura della slitta
	ed il reticolo a croce del telescopio di osservazione.
	Per effettuare tale allineamento sono state impiegate 
	le viti per effettuare spostamenti fini di cui è dotato il piatto
	rotante.
	Abbiamo assunto la lettura del goniometro
	$\alpha_0$ quale zero di riferimento per le successive misure.
	Essendo tale misura basilare per le osservazioni successive
	abbiamo iterato tale misurazione più volte
	ottenendo $\alpha_0=	\pm		$
	Effettuato questo primo step abbiamo reinserito il
	prisma orientandolo in maniera che formi un angolo 
	di $\sim 60$ gradi.
	Si è ruotato il telescopio di osservazione fino ad osservare
	tutte le righe nel visibile emesse dalla nostra sorgente;lampada al 
	cadmio.
	Ruotando a sua volte il prisma si osserva che le righe si spostano;
	si è bloccato il prisma all'angolo di inversione del moto
	della riga più prossima al giallo;ovvero la riga verde.
	Si sono adesso misurate le varie posizioni angolari 
	delle righe di emissione della lampada al cadmio
	\bigskip
	\begin{table}[hb]
		\centering
		\begin{tabular}{|c|c|c|}
 		\hline
		$\lambda [nm]$ & $\alpha\qquad(\alpha_{letto}-\alpha_{0})$ &$\Delta \alpha$ \\
  		\hline
		$467.8$  & 5 &  \\
 		$480.0$ & 19.6 &  \\
 		$508.6$ & 22.2 & \\
 		$643.8$ & 24.4 & \\
 		\hline
 		\end{tabular}
		\caption{Posizione angolari delle frange di 
 		emissione di una lampada al cadmio.Tali misure sono
 		date in riferimento all'$\alpha_0$ precedentemente determinato.\\
  		$\alpha_0=	$.}
		\label{tab:disper_angolare}
	\end{table}
	\bigskip
	.
	Effettuando un grafico della posizione angolare 
	$\alpha$ vs $1/\lambda$ si ottiene una
	dipendenza lineare con un fit a due parametri.
	%\begin{figure} [!h]
	%	\centering
	%	\includegraphics[width=0.9\textwidth]{./fit_angol}
	%	\caption{Fit della posizione angolare delle righe di emissione di 	
	%	una lampada al cadmio rispetto alla $1/\lambda$ di tali righe.}
	%	\label{fig:fit}
	%\end{figure}
	Si riporta il fit in \figura{fig:fit}.
	\smallskip

	I parametri così ottenuti
	\smallskip
	$\text{coefficiente} = a =		\pm		\text{[nm]}
	\qquad \text{intercetta} = b =		\pm		\text{[]}$
	\smallskip
	costituiscono una relazione
	tra la deviazione angolare misurata e la lunghezza d’onda osservata
	anche nella fase
	successiva.
\subsection{procedimento e misure}
	Per effettuare le misure vere e proprie abbiamo 
	rimontato la lampada al sodio.
	Dopo aver aspettato alcuni minuti,ritenuti necessari per la 
	termalizzazione
	della sorgente.
	Abbiamo osservato la posizione angolare $\alpha_{riga gialla Na}=	
	\pm		$
	Dalla relazione della calibrazione spettrale ottenuta attraverso il 
	fit 
	possiamo ricavare 
	\smallskip
	\begin{equation}
		\lambda= \frac{a}{\alpha - b} =		\pm		\text{ [nm]}
	\end{equation}
	a fronte di $ \lambda_{atteso}\sim 589 [nm]$
\section{note e osservazioni}
In questa sezione la nostra principale difficoltà
è stata associare una tolleranza alle misure fatte
per ovviare a tale ognuna delle misure fatte è
stata iterata pi volte al fine di fare statistica.
 
\part{Misura della costante di Rydberg}
\section{Scopo e strumentazione}
	L'obiettivo di questa seconda parte è di misurare la costante di Rydberg, per farlo verranno misurate le righe di emissione di una lampada all'idrogeno per mezzo di uno spettroscopio a reticolo. Preliminarmente lo strumento dovrà essere calibrato attraverso la misura della viga verde di una lampada al mercurio di lunghezza d'onda nota.
	Infine, con lo scopo di testare precisione e risoluzione dello strumento, si procederà alla misura della lunghezza d'onda del doppietto giallo di una lampada al sodio.

La strumentazione utilizzata si compone di:
\begin{itemize}
	\item uno spettroscopio a reticolo il cui goniometro ha in teoria una risoluzione di $30"$;
	\item una lampada al mercurio, una all'idrogeno e una al sodio.
\end{itemize}

Si riporta in \figurename{ \ref{fig:reticolo}} uno schema dell'apparato.
sperimentale impiegato. 
\begin{figure} [H]
	\centering
	\includegraphics[width=0.6\textwidth]{../Figs-tabs/reticolo.png}
	\caption{Schema dell'apparato impiegato nella seconda parte dell'esperienza.}
	\label{fig:reticolo}
\end{figure}

\section{Calibrazione}
\subsection{Taratura}
	Per effettare la taratra della nostra strumentazione
	abbiamo montato la lampada al mercurio,
	dopodiché abbiamo effettato n procedimento molto
	simile a qella della \sezione{sez:calib_a}.
	si é infatti
	rimosso il reticolo ed allineato la
	fenditura di uscita della sorgente,la fessura della slitta
	ed il reticolo a croce del telescopio di osservazione.
	Per effettuare tale allineamento sono state impiegate 
	le viti del piatto per effettare allineamento fine.
	Abbiamo assunto la lettura del goniometro
	$\alpha_0$ quale zero di riferimento per le successive misure.
	Essendo tale misura basilare per le osservazioni successive
	abbiamo iterato tale misurazione più volte
	ottenendo $\alpha_{ref}=	\pm		$.
	abbiamo reinserito il reticolo ponendolo in 
	maniera che esso descrivesse n angolo $\sim \ang{60}$
	tra normale e fascio incidente e che il reticolo sia allineato 
	con le lenti dei telescopi.
\subsection{Misrazione passo reticolare}
	Per effettuare la calibrazione spettrale abbiamo 
	osservato la posizione angolare per la riflessione 
	e per la riga di emissione principale del mercurio al 
	primo ordine di diffrazione $\lambda =546.074\text{ [nm]}$,
	ottenendo rispettivamente $\alpha_0=	\pm		\qquad\text{e}\qquad	\alpha_i=	\pm		$.
	Essendo nota
	\smallskip
	\begin{equation}\label{eq:passo_reticolo}
	d(sin \alpha_i - sin \alpha_d) = m \lambda\qquad \theta_i=\frac{1}{2}(\pi- \alpha_0)\qquad \alpha_d=(\pi- \theta_1-\alpha_i)
	\end{equation}
	l'\equazione{eq:passo_reticolo},dove gli angoli siano presi come in 
	\bigskip
	\begin{figure} [!h]
		\centering
		\includegraphics[width=0.9\textwidth]{./angoli.png}
		\caption{Schema della convenzioni degli angoli impiegata.}
		\label{fig:angoli}
	\end{figure}
	\smallskip
	\figura{fig:angoli},
	possiamo ricavare il passo reticolare $d$.
	Dalle misure effettate otteniamo $d=	\pm 	[mm]$

\section{Misrura $\lambda$ delle righe di emissione di na lampada al
	idrogeno}
	Per qeste misre abbiamo posto come sorgente na lampada all'idrogeno;
	dopo aver aspettato qalche minto perchè la lampada termalizzasse 
	dopodiché abbiamo allineato la lampada con la fenditra 
	posta sl telescopio di raccolta.
	Si sono misate le posizioni angolari delle varie righe osservabili
	essendo valida l'\equazione{eq:passo_reticolo}
	si ricavano le $\lambda$  riportate in 
	\smallskip
	\begin{table}[hb]
	\centering
		\begin{tabular}{|c|c|c|c|}
		\hline
		riga & ordine & $\alpha _{i}$ & $\lambda \text{ [nm]}$ \\
		\hline
		doppietto viola (intensa) & $ 1 $ &$77.66 \pm 0.05 $ & $89.1 \pm 0.1$\\
		\hline
		linea azzurra (intensa) & $ 1$ & $81.63 \pm 0.05 $ &$93.1 \pm 0.1$\\
		\hline
		doppietto verde & $ 1$ & $85.25 \pm 0.05 $ &$96.7 \pm 0.1$\\
		\hline
		linea rossa (debole) &$ 1$ & $90.66 \pm 0.05 $&$102.1 \pm 0.1$ \\
		\hline
		linea rossa (intensa) &$1$ & $93.50 \pm 0.05$ &$104.9 \pm 0.1$\\
		\hline
		doppietto viola (intensa) & $ 2$ & $ 108.42 \pm 0.05 $ &$119.9 \pm 0.1$\\
		\hline
		linea azzurra (intensa) & $ 2 $ & $115.53 \pm 0.05 $ &$127.0 \pm 0.1$\\
		\hline
		doppietto verde & $ 2$ & $122.38 \pm 0.05 $&$133.8 \pm 0.1$ \\
		\hline
		\end{tabular}
		\caption{misure delle posizioni angolari delle righe di emissione di lampada idrogeno; tali angoli sono dati rispetto ad $\alpha_{ref}$.
		Per le $\lambda$ è stata impiegata l'\equazione{eq:passo_reticolo} per n $d= 	\pm		$.}
		\label{tab:hydrogen}
	\end{table}
	\smallskip
	\tabella{tab:hydrogen}
\section{Determinazione della costante di Rydberg R}
	Impiagando l'\equazione{eq:ryd}
	\smallskip
	\begin{equation}\label{eq:ryd}
	\frac{1}{\lambda}=R(\frac{1}{({n_1}^2)}-\frac{1}{({n_2}^2)})
	\end{equation}
	\smallskip
	possiamo andare a legare la lunghezza d'onda $\lambda$
	associata alla transizione dalle righe di balmer e conoscendo 
	i relativi $n_1$ e $n_2$
	si ottengono gli $R_i$ riportati in \tabella{tab:R}.
		\smallskip
	\begin{table}[hb]
	\centering
		\begin{tabular}{|c|}
		\hline
		$ R [m^{-1}]$ \\
		\hline
		$89.1 \pm 0.1$\\
		\hline
		$93.1 \pm 0.1$\\
		\hline
		$96.7 \pm 0.1$\\
		\hline
		\end{tabular}
		\caption{$R$ misrate delle righe di emissione di lampada idrogeno.}
		\label{tab:R}
	\end{table}
	\smallskip
	Dai dati ottenti facendo n fit \bigskip
	VA FATTO FIT
	\bigskip si ottiene
	$R= 	\pm		[m^{-1}]$.
\section{Misra della risolzione dell'interferometro}
	Per effettare la misra della sensibilità dello
	spettroscopio abbiamo montato come sorgente la 
	lampada al sodio,abbiamo aspettato alcni minti
	per far termalizzare  la lampada.
	Spostato la posizione del telescopio
	di ossevazione Si sono misrate le loro posizioni angolari,
	corrispondenti  doppietto giallo
	del sodio, ottenendo $$\alpha_1=		\pm \ang{0.05}
	\text{ e }\alpha_2=		\pm \ang{0.05}$$
	impiegando l'\equazione{eq:passo_reticolo} si ottiene 
	\smallskip
	$$\lambda_1= 		\pm		\text{ [nm] e } \lambda_2= 		\pm		\text{ [nm]}$$
	Possiamo osservare che tali misure risultano ??in accordo?? con 
	la misra effettata con l'apparato in \sezione{sez:str_a}.
	Essendo nota $\Delta 	\lambda=0.6 \text{ [nm]}$
	$$HO-DBBIO-SE-NOMINALE-O-VA-MISRATO$$
	,ovvero la separazione delle de
	righe del doppietto in esame, e dalla conoscenza di 
	$\Delta \alpha= \alpha_2 -\alpha_1 =$
	possiamo effettare la vera e propria stima della
	risolzzione strmentale dal rapporto 
	$$ \frac{\Delta \lambda}{\Delta \alpha}= 	\pm			\text{ [nm]}$$
	Essendo la separazione del doppietto della riga di emissiame gialla
	della lampada maggiore della nostra risolzione strmentale.
	Pertanto la separazione del doppietto rislta osservabile.
\section{Note e considerazioni}
	Per effettare le calibrazioni degli apparati strmentali
	abbiamo dovto effettare degli allineamenti e delle 
	misrazioni angolari.
	Essendo le nostre misre soggette all'osservazione ad occhio
	per amentare la significatività di tali osservazioni abbiamo
	iterato la calibrazione più volte 
	prendendo come valore qello medio ottento.Analogo procedimento è 
	stato impiegato per ognuna delle misrazioni angolari. 
	Per la propagazione degli errori abbiamo impiegato l'usale 
	somma in quadratra.

%\input{./2b_taratura.tex}
\end{document}







