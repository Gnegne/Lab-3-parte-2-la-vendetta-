\section{Amplificatore di potenza e preamplificatore}
	Si è proceduto al montaggio del circuito in \figurename{\ref{fig:ampli}}
	impiegando le corrispondenti componentistiche.
	\begin{figure}[htb]
		\centering
		\includegraphics[scale=0.4]{ampli.png}
		\caption{Circuiti amplificatore di potenza e preamplificatore.}
		\label{fig:ampli}
	\end{figure}
	Si è proceduto successivamente alla verifica del comportamento circuitale.
	Per fare ciò si è proceduto ad acquisire la forma d'onda misurabile in $S6$
	ottenendo le acquisizioni in \figurename{\ref{fig:S6}};
	\begin{figure}[h]
		\centering
		\subfloat[S6 non schermato]{
			\includegraphics[scale=0.5]{led-rumoroso.png}
			\label{fig:S6a}
		}
		\subfloat[S6 schermato da luce ambientale ch. 2]{
			\includegraphics[scale=0.5]{s6-vuoto.png}
			\label{fig:S6b}
		}\\
		\subfloat[S6 con 2 lastre ch.2]{
			\includegraphics[scale=0.5]{s6-2-piastre.png}
			\label{fig:S6c}
		}
		\subfloat[rumore luce ambientale]{
			\includegraphics[scale=0.5]{rumore-50-hz.png}
			\label{fig:S6d}
		}\\
		\caption{Acquisizione delle forme d'onda osservate su S6}
		\label{fig:S6}
	\end{figure}
	come è possibile vedere da tali acquisizioni il segnale S6 in presenza di luce diffusa 
	nell'ambiente di lavoro presenta una forma d'onda solo approssimativamente sinusoidale (\figurename{\ref{fig:S6a}}); schermando tale segnale 
	si ottiene $V_{pp}=$\SI{2.02\pm 0.04}{\volt} ed una forma d'onda che corrisponde maggiormente ad una 
	sinusoidale \figurename{\ref{fig:S6b}}.
	Si è pertanto proceduto a inattivare il diodo LED e acquisire il segnale provocato 
	dalla luminosità 
	diffusa nel ambiente di lavoro ottenendo un segnale 
	di $V_{pp_{noise}}$ dell'ordine di \SI{400}{\milli \volt}\footnote{La $f$ rilevata $f\sim$\SI{50}{2\hertz} risulti in accordo con la frequenza 
	della tensione di rete.}; tale segnale di $noise$ risulta compatibile con i segnali 
	ottenuti frapponendo già $2$ lastre di mylar,$V_{pp}=$\SI{840 \pm 8}{\milli \volt} (\figurename{\ref{fig:S6c}}).
	A seguito di questa verifica preliminare si è stato ritenuto necessario montare il
	circuito completo in \figurename{\ref{f:complessivo} per scermare il sistema da
	i rumori ambientali.
	
	

	