%!TEX encoding = UTF-8 Unicode
%!TEX TS-program = pdflatex

%%% --- PREAMBLE --- %%%
\documentclass[a4paper,11pt]{article}

\usepackage[italian]{babel}
\usepackage[left=2cm,right=2cm,top=2cm,bottom=2cm,headheight=14pt]{geometry}
\usepackage[T1]{fontenc} % OT1: basic, T1: western, T3 and T5: exotic, T4: lots of characters but WORSE READABILITY
\usepackage[utf8x]{inputenc} % utf8x supports more characters than utf8
\usepackage{graphicx} % import PNG, JPG and PDF with \includegraphics
\usepackage[usenames,table]{xcolor} % \color
\usepackage{amssymb}
\usepackage{amsmath}
\usepackage{amsfonts}
\usepackage{float}
\usepackage{mathtools} % (!! PLACE BEFORE hyperref !!)
\usepackage{xfrac} % \sfrac
\usepackage{cancel} % \cancel \cancelto
\usepackage{hyperref} % interactive links in TOC, URLs and references
% unneded \usepackage{fixltx2e} % provides \textsubscript and makes some fixes
\usepackage[toc,page]{appendix}
\usepackage{siunitx} % \num \si \SI
\usepackage{alltt} % {alltt} (like verbatim but with commands)
\usepackage{moreverb} % {listing}
\usepackage{listings} % {lstlisting}
\usepackage[overload]{textcase} % fixes \MakeUppercase and \MakeLowercase
\usepackage[normalem]{ulem} % \uline \uwave \sout \xout
\usepackage{enumerate} % adds options for {enumerate}
\usepackage{paralist} % inline lists with {inparaenum}
\usepackage[official]{eurosym} % \euro
\usepackage{tabu} % {tabu} (like {tabular} with improvements)
\usepackage{layout} % layout description
\usepackage{multicol} % {multicols}
\usepackage{lipsum} % filling text generator with \lipsum
\usepackage[section]{placeins} % inhibits float figures from trepassing a section boundary
\usepackage{subfig} % \subfloat to be used inside {figure}
\usepackage{wrapfig} % {wrapfigure} (like {figure} but allows text to flow on its sides)
\usepackage{ifthen} % \ifthenelse
\usepackage{calc}
\usepackage{array}
\usepackage{multirow}
\usepackage{booktabs} % \toprule, \midrule, \bottomrule
\usepackage{fancyhdr}
\usepackage{wasysym}
\graphicspath{ {../Figs-Tabs/} } % graphics search directories
\setcounter{tocdepth}{1} % -1: part, 0: chapter, 1: section, 2: subsection, 3: subsubsection

\lstset{ %
	language=C,
	deletekeywords={},
	morekeywords={},
	backgroundcolor=\color{white},
	basicstyle=\ttfamily\small,
	commentstyle=\color{teal},
	keywordstyle=\color{magenta},
	stringstyle=\color{purple},
	identifierstyle=\color{violet!80!black},
	numbers=left,
	numbersep=7pt,
	numberstyle=\scriptsize\sffamily\color{gray},
	stepnumber=1,
	breakatwhitespace=false,
	breaklines=true,
	keepspaces=true,
	showspaces=false,
	showstringspaces=false,
	showtabs=false,
	tabsize=2,
	captionpos=none,
}

\newcommand{\ndr}[1]{\footnote{#1 (n.d.r.)}}

\newcommand{\fig}[1]{\figurename{ \ref{fig:#1}}} %inserting reference to figures
\newcommand{\tab}[1]{\tablename{ \ref{tab:#1}}} % inserting reference to tables
\newcommand{\eqn}[1]{equazione \eqref{eq:#1}} % inserting reference to equation

\newcommand{\dof}{\text{ dof}} % degrees of freedom
\newcommand{\paral}{\mathbin{\|}} % impedance parallel
\DeclareSIUnit\deca{decade} % decade unit definition for use in siunitx
\DeclareSIUnit\gauss{Gs} % Gauss unit definition for use in siunitx

\newcommand{\insertpart}[2]{\input{#1}}
\newcommand{\e}{\textbf{$e^{-}$}}

\sisetup{%
	separate-uncertainty = true,
	per-mode = symbol,
	bracket-numbers = false,
	multi-part-units = single,
	table-number-alignment = center,
	range-phrase = \text{--},
	range-units = single,
	output-complex-root =  \text{\ensuremath{j}},
	table-figures-decimal = 3,
	table-figures-exponent = 0,
	table-figures-integer = 2,
	table-figures-uncertainty = 2,
}

%%% --- DOCUMENT --- %%%


%%%%% SIunits example use:
% \si{\kilo\volt\per\meter\squared} -> kV/m^2
% \SI{1.222 (34)}{\joule\second}    -> 1.222 +- 0.034 Js
% \SI{1.222 \pm 0.034}{\nF}         -> 1.222 +- 0.034 nF
% use it plz

\pagestyle{fancy}
\author{Gruppo BF \\ Thomas Giannoni, Valerio Lomanto, Roberto Ribatti}
\title{Esercitazione N. 14: Misura costante di assorbimento del mylar usando un amplificatore lock-in}
\date{9 maggio 2017}

\author{Gruppo BF \\ Thomas Giannoni, Valerio Lomanto, Roberto Ribatti}
\title{Esercitazione N. 14:}
\date{9, 11 maggio 2017}

\begin{document}
\maketitle

\begin{abstract}
	L'obiettivo dell'esperienza è la misura della costante  Boltzmann tramite la misura del rumore termico prodotto da resistenze di vari valori. Per effettuare questa misura sarà necessario amplificare notevolmente il rumore generato dalla resistenza e filtrarlo su una banda di frequenze in modo da poter ricavare $k_B$ a partire dalla formula di Nyquist del rumore termico.
\end{abstract}

\section{Strumentazione}
La strumentazione usata è quella presente sul banco di lavoro, più:
	\begin{itemize}
		\item un INA114 (precision instrumentation amplifier);
		\item due IC AD708 (ultra low offset dual op-amp);
		\item un AD736 (true rms-to-dc converter).
	\end{itemize}

Essendo il circuito realizzato molto sensibile ad eventuali rumori nell'effettuare i collegamenti su breadboard si è cercato di minimizzare gli effetti di induzione e 	si è montato il circuito in \figurename{ \ref{fig:prel}}.
Si è proceduto inoltre  a alimentare tutti i componenti con le medesime linee
	di distribuzione (tensioni $V_{+}= $\SI{5.00(4)}{\volt} e $V_{-}= $\SI{-5.01(4)}{\volt}).

	\begin{figure}[h]
		\begin{minipage}{0.65\textwidth}
			\centering
			\includegraphics[scale=0.5]{prelim.png}
			\caption{Circuito di filtro per l'alimentazione.}
			\label{fig:prel}
		\end{minipage}
		\begin{minipage}{0.3\textwidth}
			\begin{tabular}{l@{ }c@{ }l}
				$R_{1}$& = &\SI{9.84(9)}{\kilo\ohm}\\
				$R_{2}$& = &\SI{9.74(9)}{\kilo\ohm}\\
				$C_1$& = &\SI{113(5)}{\nano\farad}\\
				$C_2$& = &\SI{106(5)}{\nano\farad}\\
			\end{tabular}
		\end{minipage}
	\end{figure}

\section{Metodo di misura}

	Per effettuare le misure è stato realizzato l'apparato in \figurename{ \ref{fig:completo}}
	verificando per ciascuno dei blocchi l'operatività.
	Effettuate tali verifiche sono stati collegati tra loro i vari blocchi
	e sono stati acquisiti i valori di tensione, $V_{RMS}$, letti in continua sul voltmetro,
	per vari valori di resistenze in dotazione.

	Dalla campionatura ottenuta attraverso un fit a tre parametri sono state ottenuti i valori
	di $V_{0n}$, $R_{T}$, $R_{n}$; rispettivamente : il rumore in uscita a resistenza nulla; la resistenza equivalente del rumore serie dell'amplificatore riferito all'ingresso ed il rapporto tra il rumore parallelo; il rumore serie dell’amplificatore, riferiti all’ingresso.

	\begin{figure}[h]
			\centering
			\includegraphics[scale = 0.4]{completo.png}
			\caption{schema dell'apparato di misura.}
			\label{fig:completo}
	\end{figure}

	Tali valori, essendo determinati da $k_{B}$ da relazioni descritte nel seguito, hanno permesso
	di ricavare una misura della costante di Boltzmann.

\section{Pre-amplificatore}
	Per realizzare il pre-amplificatore è stato realizzato il circuito in 
	\figurename{ \ref{fig:ampli}} impiegando le corrispondenti componenti
	circuitali.
	
		\begin{figure}[h]
		\begin{minipage}{0.75\textwidth}
			\centering
			\includegraphics[width=\textwidth]{ampli.png}
			\caption{Circuito pre-amplificatore.}
			\label{fig:pre}
		\end{minipage}
		\begin{minipage}{0.19\textwidth}
			\begin{tabular}{l@{ }c@{ }l}
				$R_{1}$& = &\SI{0.971(9)}{\kilo\ohm}\\
				$R_{2}$& = &\SI{4.69(5)}{\kilo\ohm}\\
				$R_3$& = &\SI{71.9(7)}{\kilo\ohm}\\
			\end{tabular}
		\end{minipage}
	\end{figure}
	Essendo il un guadagno atteso in $out$ $\sim 51 \cdot 14.5 \sim 740$ per effettuare
	la verifica, evitando che l'uscita dell'op-amp saturi è stato montato un partitore impiegando le resistenze $R_{T1}=$\SI{0.987(9)}{\kilo\ohm} $R_{T2}=$\SI{1031(9)}{\kilo\ohm}
	ottenendo pertanto un guadagno  $Av_{T}=0.00096 \pm 0.00001$.
	
	Si è proceduto pertanto ad inviare in ingresso al partitore un onda sinusoidale di 
	$V_{pp}=$\SI{21.0 (2)}{\volt}, ottenendo all'uscita $Ref$ un onda sinusoidale di ampiezza
	picco-picco $V_{ref}=$\SI{1.10 (1)}{\volt} (\figurename{ \ref{fig:preamp1}}).
	
		
	Il guadagno ottenuto $Av_{1}= 55 \pm 1$ risulta compatibile con le attese $\sim 51$.
	
	Effettuata questa prima verifica si è proceduto a verificare il guadagno della seconda
	parte del circuito montato, $Av_{2}= \frac{V_{out}}{V_{ref}}$.
	Essendo i due circuiti indipendenti è stato optato per non impiegare $V_{ref}$ data dall'uscita 
	del precedente op-amp
	ma un segnale sinusoidale generato esternamente, ottenendo i segnali riportati in
	\figurename{ \ref{fig:preamp2}}.
	Si ottengono per $V_{ref}=$\SI{0.374 (2)}{\volt} si ottiene un segnale in uscita
	$V_{out}=$\SI{5.68 (4)}{\volt} e pertanto un guadagno $Av_{2})=15.2 \pm 0.1$.
	Tale guadagno risulta compatibile coi valori attesi $\sim 14.5$.
	
	Il fatto che i guadagni misurati risultino maggiori di quelli previsti è stato imputato al 
	diverso valore dei resistori dai loro valori nominali.
	
		\begin{figure}[h]
		\begin{minipage}{0.45\textwidth}
			\centering
			\includegraphics[scale=0.75]{preamp1.png}
			\caption{Acquisizione segnali in ingresso al partitore ch.1 e in $ref$ ch.2.}
			\label{fig:preamp1}
		\end{minipage}
		\begin{minipage}{0.45\textwidth}
			\centering
			\includegraphics[scale=0.75]{preampb.png}
			\caption{Acquisizione segnali in ingresso ch.1 e in $out$ ch.2.}
			\label{fig:preamp2}
		\end{minipage}
	\end{figure}
	
	
\section{Squadratore e campionatore}
	Dopo aver effettuato le verifiche precedenti è stato montato il
	circuito in
	\figurename{ \ref{fig:sqd}}
	impiegando le componenti circuitali corrispondenti.

	\begin{figure}[ht]
		\centering
		\includegraphics[scale=0.35]{deriv.png}
		\caption{Circuito che svolge la funzione di squadratore e campionatore}
		\label{fig:sqd}
	\end{figure}

	Il circuito costruito svolge la funzione di squadrare e campionatore la tensioni immesse.

	Si è proceduto pertanto alla verifica del funzionamento circuitale.
	Sono state d'apprima acquisite le forme d'onda visualizzabili su $S4-S5$
	ottenendo come da attese due segnali in opposizione di fase.

	\begin{figure}[ht]
		\centering
		\includegraphics[scale=0.55]{s4-s5.png}
		\caption{segnale su s4 ch.1, s5 ch.2 }
		\label{f:s4-s5}
	\end{figure}

	Dopodiché sono stati acquisiti i segnali visualizzabili sui terminali $S7$ e $S8$
	per le due diverse posizioni del deviatore.

	Come è possibile vedere dalla \figurename{ \ref{fig:s7}} si ottiene che i due segnali, qualora il
	deviatore sia collegato ad $S2$,  risultino dovuti alle semi-onda
	positiva o negativa, a seconda della traccia in esame.

	Per il deviatore collegato ad $S1$
	si attendono invece segnali a media nulla; le tracce ottenute risultano in buon accordo
	con le attese.

	\begin{figure}[h]
		\centering
		\subfloat[Deviatore collegato ad $S1$.]{
			\includegraphics[scale=0.555]{s7-s8a.png}
			\label{fig:S7a}
		}
		\qquad
		\subfloat[Deviatore collegato ad $S2$.]{
			\includegraphics[scale=0.55]{s7-s8b.png}
			\label{fig:S7b}
		}
		\caption{Segnali acquisiti di $S7$ (ch. 1) ed $S8$ (ch. 2)}
		\label{fig:s7}
	\end{figure}

\end{document}
