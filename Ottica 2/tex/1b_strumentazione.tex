\part{Interferometro di Michelson }

\section{Strumentazione}
La strumentazione impiegata in questa sezione  si 
compone di un interferometro di Michelson e da una fotocamera,
impiegata come ausilio per il conto delle frange di interferenza.

L'interferometro di Michelson è composto da:
\begin{list}{}{}
\item \textbf{una sorgente}, costituita da un \textbf{laser HE-NE}, 
quale sorgente di $\lambda$ nota,
per la taratura del sistema; da una \textbf{lampada al mercurio} quale 
sorgente di cui analizzare la riga di emissione più intensa; una lampada
di luce bianca;
\item \textbf{un beam splitter} per dividere la radiazione della sorgente
in 2 fasci coerenti che percorreranno i
due bracci dell'interferometro;
\item \textbf{due specchi} posti al termine dei bracci 
dell'interferometro.;
\item \textbf{uno schermo} dove osservare l'interferenza alla ricombinazione dei 2 
fasci uscenti dall'interferometro e generati dal laser HE-NE.
\item \textbf{un filtro verde}, necessario per selezionare 
la riga verde dell'emissione della lampada a mercurio.
\end{list}
\bigskip

Si riporta in \figurename{ \ref{fig:schema_appar2}} uno schema dell'apparato sperimentale impiegato. 
\begin{figure} [!h]
	\centering
	\includegraphics[scale=0.5]{./pictures/immagine2}
	\caption{Schema dell'apparato impiegato per la misura.}
	\label{fig:schema_appar2}
\end{figure}
