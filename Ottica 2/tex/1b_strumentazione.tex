\newpage
\part{Interferometro di Michelson }

\section{Strumentazione}
La strumentazione impiegata in questa sezione  si 
compone di un interferometro di Michelson e da una fotocamera,
impiegata per registrare le frange di interferenza.

L'interferometro di Michelson è composto da:
\begin{list}{$\cdot$}{}
\item \textbf{una sorgente},costituita da un \textbf{laser HE-NE}, 
quale sorgente di $\lambda$ nota,
per la taratura del sistema; da una \textbf{lampada al mercurio} quale 
sorgente di cui analizzare la riga di emissione più intensa; ed infine di una lampada
di luce bianca.
\item \textbf{un beam splitter} per dividere la radiazione della sorgente
in 2 fasci coerenti che percorreranno i
due bracci dell'interferometro.
\item \textbf{due specchi} posti al termine dei bracci 
dell'interferometro. 
\item \textbf{uno schermo} ove vedere l'interferenza alla ricombinazione dei 2 
fasci uscenti dall'interferometro e generati dal laser HE-NE.
\item \textbf{un filtro verde}, impiegato nella fase di misura per selezionare 
la riga verde 
dell'emissione della lampada a mercurio.
\item \textbf{una punta} quale riferimento per il conteggio delle frange di interferenza 
della riga verde della lampada a mercurio.
\end{list}
\bigskip


\begin{figure} [!h]
	\centering
	\includegraphics[scale=0.5]{./pictures/immagine2}
	\caption{Schema dell'apparato impiegato per la misura $\Lambda$ della riga verde della lampada a mercurio.
	M1 e M2 sono i de specchi ed S è la sorgente impiegata.}
	\label{fig:schema_appar2}
\end{figure}
Si riporta in \textbf{figura \ref{fig:schema_appar2} }uno schema dell'apparato sperimentale impiegato. 