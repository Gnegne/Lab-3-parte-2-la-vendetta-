\section{note e osservazioni}
Come espresso nelle sezioni precedenti la misura richiede alcuni accorgimenti;
essendo infatti  $d\gg \lambda_{attesa}$ necessitiamo che il raggio laser abbia
incidenza  pressoché radente,
a seguito di ciò si osserva che il
punto di incidenza 
presenta un estensione non trascurabile sul calibro.
si è quindi ritenuto opportuno misurare $D_{min}$ e $ D_{max}$
assumendo come valore di $D $, il valore medio 
dell'intervallo $[D_{min};D_{max}]$  e $\Delta D$
come semi-ampiezza dell'intervallo.
Analogo procedimento è stato effettuato sugli spot
di massimo.
Abbiamo inoltre assunto che l'incertezza sul passo
reticolare $d$, per la strumentazione impiegata 
fosse trascurabile. Si è ritenuta valida tale assunzione poiché il 
calibro ventesimale viene impiegato per avere misure con una sensibilità del
$1/20$ $[mm]$ si assume pertanto che la tolleranza sulla spaziatura
$p$ sia $p<1/20[mm]$; pertanto molto minore delle altre incertezze  nelle
misure effettuate;
inoltre tale errore sarebbe essendo un errore di calibrazione;
pertanto dovrebbe essere trattato separatamente da quelli stocastici e 
escluso dagli errori impiegati nel fit lineare. 
Si segnale infine che qualora non sia indicato diversamente per la propagazione degli errori
abbiamo proceduto con l'usale somma in quadratura.