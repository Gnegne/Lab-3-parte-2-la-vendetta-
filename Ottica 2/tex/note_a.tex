\section{Note e osservazioni}
Come espresso nelle sezioni precedenti la misura richiede alcuni accorgimenti:
si osserva che il punto di incidenza 
presenta un estensione non trascurabile sul calibro.
Si è quindi ritenuto opportuno misurare $D_{min}$ e $ D_{max}$
assumendo come valore di $D$ il valore medio. Si è deciso, ai fini del fit, di considerare comunque un'incertezza di $\unit{1}{\milli\meter}$, cioè la risoluzione dello strumento poiché l'errore dovuto alla dimensione del punto di incidenza è un errore sistematico.

Lo stesso procedimento è stato usato per determinare le distanze delle frange.

Si è inoltre assunto che l'incertezza sul passo
reticolare $d$ fosse trascurabile rispetto alla precisione della misura, tale errore sarebbe comunque un errore di calibrazione pertanto dovrebbe essere trattato separatamente da quelli stocastici e 
escluso dagli errori impiegati nel fit lineare.