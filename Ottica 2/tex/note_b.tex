\section{Note e osservazioni}
In questa sezione la principale difficoltà
riscontrata in questa esperienza risulta essere il conteggio del
numero di frange di interferenza.
In caso di movimenti bruschi è facile
perdere il conto delle frange, sia in fase di 
calibrazione che in fase di misura. 

Abbiamo deciso di registrare 
il passaggio delle bande attraverso una fotocamera digitale,
avendo cura di iniziare e terminare la registrazione rispettivamente qualche 
secondo prima dell'inizio dello spostamento di M1
e qualche secondo dopo la fine del movimento.
Questo accorgimento ha permesso di poter fare il conteggio in seguito
potendo regolare la velocità del video, riducendo conseguentemente l'incertezza 
sul conteggio delle frange.

Un ulteriore accorgimento  che abbiamo usato per diminuire l'influenza 
di eventuali errori di conteggio riguarda il numero di frange che andiamo a 
considerare: si è ritenuto congruo $m> 50$
poiché al crescere di $m$ sia l'incertezza su $\Delta L$ che l'eventuale 
errore su $m$ risulta avere un influenza minore.

L'impiego della fotocamera digitale per la lampada al mercurio, e l'analisi del
relativo filmato, hanno posto in evidenza l'esistenza di un rumore di frequenza $\unit{50}{\hertz}$ o 
suoi multipli, probabilmente dovuto ad un accoppiamento con la frequenza della tensione di rete.