\section{note e osservazioni}
In questa sezione la principale difficoltà
riscontrata in questa esperienza risulta essere il conteggio del
numero di frange di interferenza.
Essendo in caso di movimenti bruschi facile
perdere il conto delle frange,sia in fase di 
calibratura che in fase di misura, abbiamo deciso di registrare 
il passaggio delle bande attraverso una fotocamera digitale,
avendo cura di iniziare e terminare la registrazione rispettivamente qualche 
secondo prima dell'inizio dello spostamento di M1
e qualche secondo dopo la fine del movimento.
Questo accorgimento ha permesso di poter fare il conteggio a seguito
potendo regolare la velocità del video;riducendo conseguentemente l'incertezza 
sul conteggio delle frange.
Un ulteriore sistema che abbiamo usato per diminuire l'influenza 
di eventuali errori di conteggio riguarda il numero di frange che andiamo a 
considerare;
Abbiamo noi ritenuto un numero congruo $m> 50$
poiché al crescere di $m$ sia l'incertezza su $\Delta L$ che l'eventuale 
errore su $m$ risulta avere un influenza minore.

Essendo il piano di lavoro poteva essere soggetto a vibrazioni
abbiamo ritenuto opportuno porre una punta,quale riferimento dietro al nostro
filtro;così da semplificare l'operazione di conteggio.

L'impiego della fotocamera digitale per la lampada al mercurio,e l'analisi del
relativo filmato, 
hanno posto in evidenza l'esistenza di un rumore di frequenza $50 Hz$ o 
suoi multipli; tale rumore spariva settando la fotocamera su ???.
Si ritiene pertanto che questo rumore sia dovuto  ad un accoppiamento con 
la tensione di rete,che in Italia è di $50 Hz$. 

Qualora non indicato diversamente anche in questa sezione le incertezze sono 
state propagate
con metodo della somma in quadratura.