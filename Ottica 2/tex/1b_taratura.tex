\section{misura $\lambda_{Hg}$ riga verde lampada a mercurio}
\subsection{taratura del sistema}
Per effettuare la misura della lunghezza d'onda 
della riga verde di emissione della lampada
a mercurio, l'apparato necessita di una fase 
preventiva di taratura.

In questa fase si cerca di porre in relazione la differenza 
di cammino ottico $\Delta x$ allo spostamento letto 
sul micrometro $\Delta L$,
ovvero il fattore di demoltiplica della leva $y$.
Per effettuare tale calibrazione abbiamo 
montato una sorgente di  $\lambda$ nota,nello specifico un laser
HE-NE di $\lambda_{He-Ne}=633 [nm]$.
Posta la scala del micrometro, in corrispondenza dello zero,
attraverso il meccanismo posto su M2 abbiamo allineato i fasci 
uscenti dai de bracci dell'interferometro;
dopodiché abbiamo preso 
come riferimento un punto e abbiamo osservato allo spostarsi dello 
specchio M1 il numero di frange che si vedono 
passare al variare di $\Delta L$.
Essendo valida \smallskip
\begin{equation}\label{eq:lambda}
2 \cdot \Delta x = m \cdot \lambda
 \end{equation}
 \smallskip
dove $m$ è il numero di frange osservate, possiamo ricavare 
il fattore di demoltiplica $y=\frac{\Delta x}{\Delta L}$.

Essendo possibile che in fase di conteggio si perdano 
alcune frange di interferenza attraverso la fotocamera 
abbiamo registrato il passaggio delle frange per effettuare 
successivamente il conteggio.
Iterato tale conteggio più volte abbiamo trovato \smallskip
$m=73 \pm 1 \qquad \text{corrisponde a}\qquad \Delta L =120\pm 1 [\mu m]$
\smallskip
se ne ricava pertanto come valore di $y= 0.193	\pm	3	$ il valore di m è
stato dato con un incertezza poiché risultava comunque individuare l'effettivo 
passaggio dell'ultima frangia.
\subsection{misure e procedimento}
Per effettuare la misurazione di $\lambda_{Mg}$
abbiamo sostituito il laser HE-NE con 
lampada a mercurio.
Essendo lo spettro di emissione della lampada 
composto di numerose righe
abbiamo frapposto un filtro tra la sorgente e il beam splitter,
così da osservare esclusivamente la riga verde,la più intensa.
Essendo la nostra radiazione non abbastanza intensa da essere 
proiettata sullo schermo il conteggio delle frange di interferenza
è stata effettuata registrando direttamente dall'uscita del beam-splitter; 
per semplificare il conteggio abbiamo 
posto come riferimento una punta sul nostro filtro ed effettuato il conteggio 
dalla registrazione come nella nella fase di calibrazione.
Osservando il $\Delta L$ corrispondenti a $m$ frange di
interferenza applicando l'\equazione{eq:lambda} e dalla conoscenza di $y$ 
possiamo ottenere $\lambda$;
nella fattispecie avendo osservato che \smallskip
 $m=71 \pm 1\text{corrisponde a}\qquad \Delta L =10\pm 1 [\mu m]$
 \smallskip
impiegando l'\equazione{eq:lambda} si ottiene $\lambda_{Hg}=544\pm 5$ che 
risulta essere in accordo col valore atteso 	
$\lambda_{attesa}\sim 546 [nm]$
\section{Frange di interferenza luce bianca}
Per osservare le frange di interferenza con la luce bianca
abbiamo inizialmente posto uno spessore metallico su M1.
Abbiamo allineato nuovamente il sistema impiegando il laser;
dopodiché spostando M1 abbiamo ritrovato le frange di interferenza
per la lampada a mercurio e il filtro verde; così da rendere 
la differenza tra i due bracci dell'interferometro quasi nulla.
Spostando ulteriormente M1 abbiamo ottenuto le frange di interferenza per
la luce bianca,si riporta l'osservazione in 
\bigskip
\begin{figure} [!h]
	\centering
	\includegraphics[width=0.9\textwidth]{./pictures/frange.jpg}
	\caption{frange di interferenza per luce bianca.}
	\label{fig:frangeb}
\end{figure}
\bigskip
\figura{fig:frangeb}.
Una possibile giustificazione del perché il cammino ottico sui
de bracci dell'interferometro debba essere all'incirca uguale può
essere il fatto che in tale configurazione l'interferenza non risulta
dipendente da $\lambda$.  