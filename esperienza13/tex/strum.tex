\section{Strumentazione}
	In questa esperienza sono state impiegati:
	\begin{itemize}
		\item alcuni circuiti integrati:
		\begin{enumerate}
			\item 1 IC SN74LS00 (Quad NAND Gate);
			\item 2 IC SN74LS74 (Dual D-Latch);
			\item 1 IC SN74LS08 (Quad AND Gate);
			\item 1 IC SN74LS2 (Quad OR Gate);
		\end{enumerate}
		\item 1 Switch a 4 bit;
		\item 3 diodi LED; rispettivamente verde giallo e rosso
		\item il generatore di onde quadre.
		\item un microcontrollore ARDUINO NANO
		\item un oscilloscopio digitale
	\end{itemize}
\section{Semafori con circuiti integrati }
Il semaforo può presentare la modalità ABILITATO e DISABILITATO. Qualora sia abilitato si deve ottenere ciclicamente
\begin{center}
	Verde acceso $\longrightarrow$ Verde e
	Giallo acceso $\longrightarrow$ Rosso acceso $\longrightarrow$
	Verde acceso
\end{center}
Mentre qualora il semaforo sia disabilitato si deve ottenere
\begin{center}
	LED Giallo acceso $\leftrightarrow$ Tutti i LED spenti
\end{center}

Per descriminare tra i due modalità di funzionamento può essere impiegato un segnale di abilitazione
ENABLE (En).
Per la realizzazione del semaforo completo è stato realizzato d'apprima un semaforo privo di enable , per cui il sistema risulti costantemente abilitato;
dopodiché riprendendone lo schema circuitale si è introdotto il segnale di ENABLE.
