\section{Strumentazione}
La strumentazione usata è quella presente sul banco di lavoro, più:
	\begin{itemize}
		\item alcuni circuiti integrati:
		\begin{enumerate}
			\item 1 IC SN74LS00 (Quad NAND Gate);
			\item 2 IC SN74LS74 (Dual D-Latch);
			\item 1 IC SN74LS08 (Quad AND Gate);
			\item 1 IC SN74LS2 (Quad OR Gate);
		\end{enumerate}
		\item 1 Switch a 4 bit;
		\item 3 diodi LED; rispettivamente verde giallo e rosso;
		\item il microcontrollore Arduino Nano.
	\end{itemize}
\section{Semafori con circuiti integrati }
Il semaforo può presentare la modalità ABILITATO e DISABILITATO. Qualora sia abilitato si deve 
ottenere Verde acceso $\longrightarrow$ Verde e 
giallo acceso $\longrightarrow$ rosso acceso $\longrightarrow$ 
verde acceso  ciclicamente. Mentre qualora il semaforo sia disabilitato si deve ottenere
Led giallo acceso $\leftrightarrow$ Led giallo spento.
Per descriminare tra i due modalità di funzionamento può essere impiegato un segnale di abilitazione 
ENABLE (En).
Per la realizzazione del semaforo completo è stato realizzato d'apprima un semaforo privo di enable , per cui il sistema risulti costantemente abilitato;
dopodiché riprendendone lo schema circuitale si è introdotto il segnale di ENABLE.