\section{Implementazione della FSM su Arduino}

Si è proceduto a scrivere il codice per Ardunio per implementare una macchina a stati finiti e se ne è verificato il funzionamento.
Il codice è riportato di seguito.

\begin{verbatim}
#define INP      12    // pin enable
#define LED1     9     // pin green led  
#define LED2     10    // pin yellow led
#define LED3     11    // pin red led

#define timeg    10   // durata led verde in periodi di clock
#define timey    3    // durata led verde/giallo insieme in periodi di clock
#define timer    10   // durata led rede in periodi di clock
#define timeen   1    // durata led giallo e tutto spento in periodi di clock
#define clockk   500  // periodo clock in ms

/*
state 0 => LED verde
state 1 => LED verde e giallo insieme
state 2 => LED rosso
state 3 => LED giallo
 state 4 => spento
*/

int LED[3] = {LED1, LED2, LED3};
int time_state[5] = {timeg, timey, timer, timeen, timeen};
int state = 0;
int out[5][3] = {{1,0,0},{1,1,0},{0,0,1},{0,1,0},{0,0,0}};

int tnow = 0;   // tempo attuale
int tprec = 0;  // tempo dall'ultima scatto del clock
int tstate = 0; // tempo dall'ultimo cambio di stato
bool enable = HIGH;


void setup(){
	pinMode(LED1, OUTPUT);
	pinMode(LED2, OUTPUT);
	pinMode(LED3, OUTPUT);
	pinMode(INP, INPUT_PULLUP);
}

void loop(){
	tnow = millis();
	if(ClockEnd(tnow, tprec)){ // verifica se è passato un periodo di clock
		enable = digitalRead(INP);  // legge lo stato dell'enable
		if (EndState(state, tnow, tstate)){ // verifica se lo stato attuale è terminato
			state = NextState(state, enable); // aggiorna lo stato calcolando il successivo
			Output(state);  // genera l'output
			tstate = tnow;  // salva il momento di inizio dello stato attuale
		}
		tprec=tnow; // salva il momento di scatto del clock
	}
}

bool ClockEnd(int tnow, int tprec){ // TRUE se è passato un periodo di clock
	return tnow-tprec > clockk;
}

bool EndState(int state, int tnow, int tstate){ // TRUE se lo stato attuale è terminato
	return tnow-tstate > time_state[state]*clockk;
}

int NextState(int state, bool enable){ // calcola il prossimo stato 
	if(enable) return (state+1)%3;
	else  return state%2 + 3;
}

void Output(int stato){ // genera l'output dallo stato
	for (int j=0; j<3; j++){
		digitalWrite(LED[j], out[state][j]);
	}
}	
\end{verbatim}